\chapter{Hardware}
	\section{Relationship between Hardware and Software}
	
	\section{Main Memory}
	
		\subsection{Stored Program Concept}
			What separates computers from other machines is their ability to perform an infinite array of tasks, from interactive games, to word processing and performaing scientific calculations, all without needing change hardware . A washing machine contains a basic processor to control the different cycles and temperatures, but trying to get it to perform even a basic calculation you will struggle. This flexibility comes down to a concept known as the \index{Stored Program} \textit{Stored Program} concept. The idea behind it is remarkably simple, you have a fixed hardware platform capable of executing a fixed repertoire of instructions. At the same time these instructions can be used and combined to create arbitrary sophisticated programs. Moreover, the logic of these programs are not embedded in hardware, as it was in early computers, instead they are stored and manipulated in the computer's memory.   
		\subsection{Addressable Memory}
		
	\section{Secondary Storage}
		
		\subsection{Hard Disk Drive}
		
		\subsection{Solid State Drive}
			
			\subsubsection{NAND Flash Memory}
			
			\subsubsection{SSD Controller}
			
			\subsubsection{Pages}
			
		\subsection{Optical Drive}
		
	