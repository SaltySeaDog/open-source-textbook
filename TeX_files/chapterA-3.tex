% !TeX root = ../main.tex
\chapter{Introduction to C}

One of the most popular languages ever developed is called C. It was created by a group including Dennis Ritchie and Brian Kernighan at Bell Laboratories between 1969 and 1973 to rewrite the UNIX operating system from its original assembly language. By many measures, C \footnote{including a family of closely related languages such as C\#, C++ and Objective-C} is the most widely used language in existence. Its popularity stems from a number of factors:
\begin{itemize}
	\item Available on a tremendous variety of platforms, from super computers down to embedded microcontrollers.
	\item Relative Ease of use.
	\item Moderate level of abstraction providing higher productivity that assembly language, yet giving the programmer a good understanding of how the code will be execute.
	\item Suitability for generating high performance programs.
	\item Ability to interact directly with the hardware.
\end{itemize} 

	\section{My First Program}
		
		A C program is a plain text file that describes operations for the computer to perform. The text file is then \textit{compiled}, converting into from the human readable format that is was written in to the machine readable format that is its needed to be executed. C programs are generally contained in one or more text files that end with the extention \enquote{.c}. 
		
		\begin{listing}[h]
			\centering
			\begin{ccode}
#include<stdio.h>

int main(void) {
	printf("Hello World!\n");
}
			\end{ccode}
			Which Outputs:
			\begin{minted}{bash}
$~ Hello World!
			\end{minted}
			\caption{\label{code:myfirstprogram} Our first see program}
		\end{listing}
	
		In general, a C program is organised into one or more functions. Every program must include the \cinline{main} function, which is where the program starts executing. Most programs use other functions defined elsewhere in the code and/or in in a library. We can split our code in snippet \ref{code:myfirstprogram} into the header, the \cinline{main} function, and the body.
		
		
		